\input{text/diss}
\begin{document}
\def\labauthors{Войтович Д.А., Понур К.А.}
\def\labgroup{440}
\def\labnumber{2}
\def\labtheme{Замедляющие системы типа гребенки}
\def\department{Кафедра электродинамики}
\input{text/titlepage}
\newpage
\paragraph{Цель работы.} 

Цель настоящей работы состоит в изучении волн, направляемых замедляющими системами типа гребенок. Общее описание таких систем весьма непросто, поэтому мы здесь ограничимся частным случаем, допускающим использование понятия поверхностного импеданса. Сначала мы обсудим характеристики волн, направляемых плоскостью с заданным импедансом, а затем уже --
конкретную реализацию импеданса в гребенчатых структурах.

\section{Теоретическая часть}
\subsection{Введение}



\section{Практическая часть} % (fold)
\subsection*{Задание 1. Дисперсионные характеристики гребенок.} % (fold)
Были сняты дисперсионные характеристики двух гребенок, различающихся высотой зубьев: $l_1 = 8$ мм и $l_2 = 22 $ мм.
Зависимость частоты  $\nu$ от продольного волнового числа  $h$ представлена на рис.\ref{fig:1}.
\begin{figure}[h!]
\centering
\includegraphics[width=0.9\linewidth]{rec/task1.pdf}
\caption{Дисперсионные характеристики гребенок}
\label{fig:1}
\end{figure}

\subsection*{Задание 2. Продольное распределение поля.} % (fold)
На гребенке 1 было прослежено изменение характера распределения поля вдоль системы с изменением частоты в широких пределах вплоть до частоты запирания: $\omega_{\text{зап}}\sim \frac{c \pi}{2l}$ 
На основе экспериментальных данных (см. рис.\ref{fig:2} ) можно сделать вывод о частоте запирания данной гребенки $\omega_{\text{зап}}\simeq 3100 $ ГГц. 
\begin{figure}[h!]
    \centering
    \includegraphics[width=0.9\linewidth]{rec/task2.pdf}
    \caption{Распределение поля вдоль системы}
    \label{fig:2}
\end{figure}

\subsection*{Задание 3. Зависимость поля от положения крышки гребенки.} % (fold)

\end{document}
